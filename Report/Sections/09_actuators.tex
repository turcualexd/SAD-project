\section{Actuators}
\label{sec:actuators}


\subsection{Magnetorquers}
\label{subsec:magnetorquers}

Magnetorquers are actuators capable of inducing a torque on the spacecraft through the generation of a magnetic dipole, according to \autoref{eq:mag_torque}. This dipole can be generated by an electrical current flowing into a coil, as stated in the Ampere-Maxwell equation from electromagnetism. To be capable of generating a $\boldsymbol{D}$ vector in any direction of 3D space, three coils have to be placed on three perpendicular axis in order to generate three indipendent components of the magnetic dipole.

Magnetorquers are probably the most used attitude control actuators nowadays. This is primarly due to their extreme versatility and inexpensiveness for little satellites orbiting sufficiently near to Earth. All this combined with their virtually unlimited lifetime (because they only need current which can come from solar energy) give the magnetorquer a wide range of applicability. This comes clearly with some downsides. The most limiting factor for this type of actuator is the low torque generated, typically in the range of $10^{-3} \div 10^{-6}$ Nm, that could render the control action really slow, expecially for more big and heavy satellites. 

Another limiting factor is of course the fact that they rely completely on the external magnetic field, which could vary a lot during the orbit and there could be zones in which the effectiveness is greatly reduced due to a weak $\boldsymbol{B}$ vector. Furthermore, the satellite could be a source of disturbance due to some parasite currents flowing in other electronic devices. The same could be said for the magnetic coils themselves, which can be source of disturbance for the magnetic sensors.

In addition of all this, the torque generated by these actuators cannot have three indipendent components, since the torque can be generated only in a plane perpendicular to the $\boldsymbol{B}$ of the magnetic field. This can be demonstrated from the \autoref{eq:mag_torque}, which is reported and expanded below:

\begin{equation}
    \boldsymbol{M} = \boldsymbol{D} \times \boldsymbol{B} =
	\left[ -\boldsymbol{B} \times \right] \boldsymbol{D}
	\implies
	\begin{bmatrix}
		M_x \\ M_y \\ M_z
	\end{bmatrix}
	=
	\begin{bmatrix}
		0 & B_z & -B_y \\
		-B_z & 0 & B_x \\
		B_y & -B_x & 0
	\end{bmatrix}
	\begin{bmatrix}
		D_x \\ D_y \\ D_z
	\end{bmatrix}
\end{equation}

\begin{equation} \label{eq:dipole}
	\boldsymbol{D} = \left[ -\boldsymbol{B} \times \right] ^{-1} \boldsymbol{M}
\end{equation}

Since $\left[ -\boldsymbol{B} \times \right]$ is a singular matrix, it cannot be inverted as shown in \autoref{eq:dipole}, so the complete controllability of the system through magnetic actuators is not assured.
For this reason, as the satellite is not GG stable (as shown in \autoref{subsec:dist_GG}) and the chosen orbit presents important variations of the $\boldsymbol{B}$ vector during the simulation, the actuation has been integrated with two reaction wheels (more details in \autoref{subsubsec:act_cmd_logic}).

The magnetorquer that was selected for this project is the \textit{MTQ800} from \textit{ACC Clyde Space}, which details are reported below \cite{magnetorquer_site}:
\begin{table}[H]

    \centering
    \begin{tabular}{|c|c|}
    \hline
    $\bm{D_{max} \, [Am^2]}$ & $\bm{frequency \, [Hz]}$ \\
    \hline
    $15$ & $30$ \\
    \hline
    \end{tabular}
    
    \caption{Real data for Magnetorquer}
    \label{table:magnetorquer}
    
\end{table}

The Simulink model takes as input the ideal dipole requested by the controller, elaborates it through a rate limiter and a saturation limiter, then it introduces some random white noise. This model, which has been used also for the other actuators, is designed to replicate the accuracy, the dynamics and the limits of the actuators in the real world. Otherwise, the control would be ideal and it could generate some torques that would be impossible to obtain with physical actuators.


\subsection{Reaction Wheels}
\label{subsec:reaction_wheels}

Reaction wheels are one of the primary attitude control actuators for controlling the spacecraft, as they are reliable and can give precise pointing. Their working principle is based on the internal angular momentum exchange between the satellite and the reaction wheel itself. At every time instant, the  angular momentum of the overall system composed by the reaction wheel and the rest of the satellite can be expressed as:

\begin{equation}
	\boldsymbol{h}=\boldsymbol{I}\boldsymbol{\omega}+\boldsymbol{h_r}
\end{equation}

where $\boldsymbol{h_r}$ is the angular momentum of the reaction wheel with respect to the rest of the spacecraft along its spin axis.
If no external torque is applied, due to the conservation of angular momentum, it can be written:

\begin{equation}
	\boldsymbol{I}\boldsymbol{\dot{\omega}}=-\boldsymbol{\dot{h}_r}
\end{equation}

Usually $\boldsymbol{\dot{h}_r}$ is due to the torque applied to the reaction wheel by an electric motor, that has its proper operational curve. The latter can be modeled as constant until a certain angular velocity $\boldsymbol{\omega_{max}}$ is reached by the reaction wheel. When the saturation speed $\boldsymbol{\omega_{max}}$ is achieved, the reaction wheel cannot provide any additional torque, either positive or negative. If the torque required is more or less periodic, it is possible to choose the actuator in order to have it always in the functioning regime. If, in addition to the periodical component, there is a secular contribution, then it is inevitable to reach the saturation speed. To avoid this, it is necessary a desaturation maneuver, executed with the help of the magnetorquers.

The model chosen for the reaction wheels is the \textit{OCE-RW40}, which is characterized by the following specifications \cite{reactionwheel_site}:

\begin{table}[H]

    \centering
    \begin{tabular}{|c|c|c|c|}
    \hline
    $\bm{\omega_{max} \, [RPM]}$ & $\bm{\dot{h}_{max} \, [Nm]}$ & $\bm{\sigma_{\omega} \, [RPM]}$ & $\bm{frequency \, [Hz]}$\\
    \hline
    $6000$ & $0.1$ & $2$ & $30$ \\
    \hline
    \end{tabular}
    
    \caption{Real data for Reaction Wheel}
    \label{table:RW}
    
\end{table}

The Simulink model of the reaction wheel takes as input the ideal torque. The $\boldsymbol{\dot{\omega}_r}$ is computed dividing by the inertia of the wheel around its spin axis. $\boldsymbol{\omega_r}$ is computed integrating $\boldsymbol{\dot{\omega}_r}$. $-\boldsymbol{\dot{h}_r}$, which comes from the control block, is multiplied by the result of the boolean operation $\boldsymbol{\omega_r}\le\boldsymbol{\omega_{max}}$. At this point, the signal passes through a rate limiter and  a saturator. It is added a band limited white noise.