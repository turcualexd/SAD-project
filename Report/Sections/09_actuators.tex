\section{Actuators}
\label{sec:actuators}


\subsection{Magnetorquers}
\label{subsec:magnetorquers}

Magnetorquers are actuators capable of inducing a torque on the spacecraft through the generation of a magnetic dipole, according to \autoref{eq:mag_torque}. This dipole can be generated by an electrical current flowing into a coil, as stated in the Ampere-Maxwell equation from electromagnetism. To be capable of generating a $\boldsymbol{D}$ vector in any direction of 3D space, three coils have to be placed on three perpendicular axis in order to generate three indipendent components of the magnetic dipole.

Magnetorquers are probably the most used attitude control actuators nowadays. This is primarly due to their extreme versatility and inexpensiveness for little satellites orbiting sufficiently near to Earth. All this combined with their virtually unlimited lifetime (because they only need current which can come from solar energy) give the magnetorquer a wide range of applicability. This comes clearly with some downsides. The most limiting factor for this type of actuator is the low torque generated, typically in the range of $10^{-3} \div 10^{-6}$ Nm, that could render the control action really slow, expecially for more big and heavy satellites. 

Another limiting factor is of course the fact that they rely completely on the external magnetic field, which could vary a lot during the orbit and there could be zones in which the effectiveness is greatly reduced due to a weak $\boldsymbol{B}$ vector. Furthermore, the satellite could be a source of disturbance due to some parasite currents flowing in other electronic devices. The same could be said for the magnetic coils themselves, which can be source of disturbance for the magnetic sensors.

In addition of all this, the torque generated by these actuators cannot have three indipendent components, since the torque can be generated only in a plane perpendicular to the $\boldsymbol{B}$ of the magnetic field. This can be demonstrated from the \autoref{eq:mag_torque}, which is reported and expanded below:

\begin{equation}
    \boldsymbol{M} = \boldsymbol{D} \times \boldsymbol{B} =
	\left[ -\boldsymbol{B} \times \right] \boldsymbol{D}
	\implies
	\begin{bmatrix}
		M_x \\ M_y \\ M_z
	\end{bmatrix}
	=
	\begin{bmatrix}
		0 & B_z & -B_y \\
		-B_z & 0 & B_x \\
		B_y & -B_x & 0
	\end{bmatrix}
	\begin{bmatrix}
		D_x \\ D_y \\ D_z
	\end{bmatrix}
\end{equation}

\begin{equation} \label{eq:dipole}
	\boldsymbol{D} = \left[ -\boldsymbol{B} \times \right] ^{-1} \boldsymbol{M}
\end{equation}

Since $\left[ -\boldsymbol{B} \times \right]$ is a singular matrix, it cannot be inverted as shown in \autoref{eq:dipole}, so the complete controllability of the system through magnetic actuators is not assured.
For this reason, as the satellite is not GG stable (as shown in \autoref{subsec:dist_GG}) and the chosen orbit presents important variations of the $\boldsymbol{B}$ vector during the simulation, the actuation has been integrated with two reaction wheels (more details in \autoref{subsubsec:act_cmd_logic}).


\subsection{Reaction Wheels}
\label{subsec:reaction_wheels}

Reaction wheels are one of the primary attitude control actuators for controlling the spacecraft, as they can give precise pointing and be controlled through feedback actively. They are based on acceleration and deceleration of spinning rotors. If the nominal condition is with a null angular velocity then that is a defining characteristic for reaction wheels as actuator.

Now, the equations for a dual spin satellite with no external torques applied are:
     \begin{align*}
		  I_{r}\dot{\omega}_{r} &= M_{r}
     \end{align*}
     \begin{align*}
	    I_{z}\dot{\omega}_{z} &= -I_{r}\dot{\omega}_{r}
     \end{align*}
The torque \( M_{r} \) is generated by an electric motor, which follows a characteristic performance curve. The general problem of control with reaction wheels can be modelled through the adequate set of Euler equations which contains the effects of the \( n \) actuators in the expression of the angular momentum:

    \begin{equation*}
		\mathbf{h} = \boldsymbol{I} \boldsymbol{\omega} + \Delta \mathbf{h}_r
    \end{equation*}	
To simplify the solution of the control problem, considering the dynamics following the previous expression, we group the equation terms for determining the control torque required and the the effective control command for the reaction wheels:

	\begin{align*}
	\boldsymbol{I} \dot{\boldsymbol{\omega}} + \boldsymbol{\omega} \times \boldsymbol{I} \boldsymbol{\omega} + \boldsymbol{\omega} \times \Delta \mathbf{h}_r + \Delta \dot{\mathbf{h}}_r &= \mathbf{T} 
	\end{align*}
	\begin{align*}
		\mathbf{M}_c &= -\boldsymbol{\omega} \times \Delta \mathbf{h}_r - \Delta \dot{\mathbf{h}}_r
	\end{align*}
	\begin{align*}
			I_r \dot{{\omega}} &= {M}_r 
	\end{align*}
The equation for the evaluation of the control law, hence, stands as:
	
	\begin{equation*}
		\boldsymbol{I} \dot{\boldsymbol{\omega}} + \boldsymbol{\omega} \times I \boldsymbol{\omega} = \mathbf{T} + \mathbf{M}_c
	\end{equation*}\\
Once \( \mathbf{M}_c \) is calculated using any suitable linear control design technique,  \( \mathbf{M}_r \) can be evaluated using the following procedure:
	
	\begin{align*}
		\Delta \mathbf{h}_r &= -\mathbf{M}_c - \boldsymbol{\omega} \times \Delta \mathbf{h}_r 
	\end{align*}
	\begin{align*}
	\dot{\mathbf{h}}_r &= -\mathbf{A}^* ( \mathbf{M}_c + \boldsymbol{\omega} \times \Delta \mathbf{h}_r )
    \end{align*}
Here \( \mathbf{A}^* \) is the pseudo inverse matrix of \( \mathbf{A} \) for the general case.

The Simulink implementation of the reaction wheel is performed by an input signal that is the required torque that the actuators should theoretically give. As previously mentioned, the reaction wheel must produce an angular acceleration to generate that torque on the satellite. Since the reaction wheel are physical objects, there are some saturation constraints on them. In particular, the producers specifies a maximum speed that can be reached and also a maximum torque that can be generated. Also, the actuator is affected by noise, for the analyzed case modelled as a band limited white noise. 

The model chosen for the reaction wheels is the \textit{OCE-RW40}, which is characterized by the following specifications:

\begin{table}[H]

    \centering
    \begin{tabular}{|c|c|c|}
    \hline
    $\bm{\omega_{max} \, [RPM]}$ & $\bm{\dot{h}_{max} \, [Nm]}$ & $\bm{\sigma_{\omega} \, [RPM]}$ \\
    \hline
    $6000$ & $0.1$ & $2$  \\
    \hline
    \end{tabular}
    
    \caption{Real data for Reaction Wheel}
    \label{table:RW}
    
\end{table}

All these real effects are modelled in the Simulink block of the reaction wheel. 
	
