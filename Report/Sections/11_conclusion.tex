\section{Conclusion}
\label{sec:conclusion}

In the present section, the results of the simulation will be presented and discussed. The simulation has been executed for a period of two complete orbits in order to observe all the three main phases of the satellite's mission. It has been started from random initial conditions on both angular velocity $\boldsymbol{\omega}$ and Euler angles $\boldsymbol{s}$.
The $\boldsymbol{\omega}$ components were limited in range $-0.06 \div 0.06 \; \text{rad/s}$, while $\boldsymbol{s}$ components were in range $0 \div 2 \pi \; \text{rad}$.

Since the final mission of the satellite is Nadir pointing, particular attention is posed on the values of some physical quantities, in particular:

\begin{itemize}[wide,itemsep=3pt,topsep=3pt]
    \item $\boldsymbol{\omega}$ and $\lVert \boldsymbol{\dot{\hat{{B}}}_m} \rVert$ for the detumbling phase;
    \item $\boldsymbol{\alpha}$ and $\boldsymbol{\dot{\alpha}}$ for the slew and tracking phase.
\end{itemize}

Furthermore, from the satellite operational point of view, some other physical quantities are relevant, such as the measured magnetic field $\boldsymbol{B_m}$ and the computed values like the evaluated attitude matrix $\boldsymbol{A_{BN,sens}}$ and the control torques $\boldsymbol{M_c}$. In addition, the angular velocities of the reaction wheels and the magnetorquers' generated dipoles will be analyzed to verify the saturation condition and the consumption power respectively.


\subsection{Detumbling analysis}
\label{subsec:detumb_analysis}

\twofigII{omega_true.eps}{$\boldsymbol{\omega}$ in $\mathcal{B}$ frame}{omega_true}{Bm_dot.eps}{$\lVert \boldsymbol{\dot{\hat{{B}}}_m} \rVert$ in $\mathcal{B}$ frame}{Bm_dot}{1}

As it can be seen in \autoref{fig:omega_true} and \autoref{fig:Bm_dot}, the detumbling has a duration of less than half orbit, then the slew and tracking control engages to keep the angular velocity stabilized. Note that from this particular case nothing general can be deduced, as the period and the behaviour of the detumbling phase strongly depends on the initial conditions of the satellite and also on the condition on which the control switch to the next phase. In this simulation, this criteria is based on the evaluation of $\lVert \boldsymbol{\dot{\hat{{B}}}_m} \rVert$.
Looking at the graph in \autoref{fig:Bm_dot}, the detumbling phase ends when the condition $\lVert \boldsymbol{\dot{\hat{{B}}}_m} \rVert < 0.003 \; \text{s}^{-1}$ is satisfied.

\subsection{Slew and tracking phase analysis}
\label{subsec:slew_analysis}

\twofigII{alpha.eps}{Attitude error $\boldsymbol{\alpha}$ (\autoref{eq:alpha})}{alpha}{alpha_zoom.eps}{Zoom on $\boldsymbol{\alpha}$}{alpha_zoom}{1}

\twofigII{alpha_dot.eps}{Angular velocity error $\boldsymbol{\dot{\alpha}}$ (\autoref{eq:alpha_dot})}{alpha_dot}{alpha_dot_zoom.eps}{Zoom on $\boldsymbol{\dot{\alpha}}$}{alpha_dot_zoom}{1}

In \autoref{fig:alpha} and \autoref{fig:alpha_dot} it can be seen that, after the detumbling, there is a transient after which both attitude and angular velocity errors converge to zero in a short period of time and keep stabilized.
To better evaluate the errors in the tracking phase, in \autoref{fig:alpha_zoom} and \autoref{fig:alpha_dot_zoom} a zoom on the final phase can be appreciated. The high frequency oscillations are mainly imputated to errors introduced by sensors and actuators. Indeed, both the command to the actuators and the executed control action are affected by noise.