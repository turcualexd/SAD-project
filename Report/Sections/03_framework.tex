\section{Framework Analysis}
\label{sec:framework}

\subsection{Satellite characterization }
\label{subsec:sat_characterization}



\subsection{Orbit characterization}
\label{subsec:orbit_characterization}

The orbit adopted for the simulation is a Sun-synchronous (SSO), nearly polar and LEO orbit. Polar orbits allows to scan the whole globe during
the several orbits, due to Earth rotation. SSO are orbits that maintain the same angle between their orbital plane and the direction that connects
the Earth with the Sun \cite{curtis_book}. This allows the spacecraft the monitor the Earth surface with always the same conditions of light 
(or eventually darkness, if the plane is oriented in a certain way). Also, a SSO orbit can be choosen in such a way to have always the sun visible \cite{esa_sso_site}.

The real data are based on the ephemeris of the ESAIL mission from which we were inspired. In particular, it was taken the orbital parameters on 16/12/2023 at $12$ UT of ESAIL satellite, then we 
propagated the orbit using the simple two body problem  without any perturbation. This clearly is an approximation since several distrubances act 
on the satellite as it will be seen in \autoref{sec:disturbances_analysis}, also the SSO orbits are intrisically caused by the J2 effect of Earth.
Nevertheless, the simulation of few LEO orbit's periods considered in this report wouldn't be enough to show the distrubances effects caused on 
the motion of the centre of mass of the satellite. The advantage to take as initial condition the ephemeris is that the motion of the spacecraft in 
those two or three periods of the orbit of simulation that are considered is seen as sun-synchronous. Infact, that time of simulation taken into account
is a snapshot compared to the time of action of the J2 effect responsible for the SSO orbit, that is one year. Clearly, a more detailed simulation should
consider the variation of the orbital parameters due to J2 and all other perturbations.

The orbital parameters chosen, following the description given above, are:

\begin{table}[H]

    \centering
    \begin{tabular}{|c|c|c|c|c|}
    \hline
    $\bm{a \, [km]}$ & $\bm{e \, [-]}$ & $\bm{i \, [\deg]}$  & $\bm{\omega \, [\deg]}$   & $\bm{\Omega \, [\deg]}$ \\
    \hline
    $6851$ & $0.0018$ & $97.40$ & $101.58$ & $0$ \\
    \hline
    \end{tabular}
    
    \caption{Orbital Parameters}
    \label{table:orb_table}
    
\end{table}

\twofigII{orbit.eps}{Orbit Representation}{leo_orbit}{orbit_2.eps}{Sun Direction view}{leo_orbit2}{1}

In the \autoref{fig:leo_orbit} and \autoref{fig:leo_orbit2} the sun direction is also plotted, in this case it is possible
to see that the orbit doesn't go into eclipse condition.

On Simulink, the model for the orbital position implemented is based on the integration of the true anomaly

\begin{empheq}{equation*}
   \dot{\theta} = \frac{n \left(1 + e\cos{\theta}\right)^2}{\left(1 - e^2 \right)^{3/2}}
\end{empheq}

Then, the radial distance is found as:

\begin{empheq}{equation*}
    r = \frac{a \left(1 - e^2 \right)}{1 + e\cos{\theta}}
\end{empheq}

 At this point it is easy to retrieve the position $\boldsymbol{r_p}$ of the S/C in the perifocal frame $\mathcal{P}$.

 \begin{align}
    \underline{r}_{P} &= r\begin{bmatrix}
           cos \theta \\
           sin \theta \\
           0
         \end{bmatrix}
  \end{align}.

  The position in the inertial frame $\mathcal{N}$ is found using the  the transpose of the transformation matrix 
  $\boldsymbol{A_{pn}} = \boldsymbol{R_3}\left(\omega\right) \boldsymbol{R_1}\left(i\right) \boldsymbol{R_3}\left(\Omega\right)$. 
  In particular:
  \begin{empheq}{equation*}
    \boldsymbol{r_n} = \boldsymbol{A_{pn}r_p} 
 \end{empheq}