\section{Environment}
\label{sec:environment}

\subsection{Satellite characterization }
\label{subsec:sat_characterization}


\subsection{Orbit characterization}
\label{subsec:orbit_characterization}

\subsection{Disturbances analysis}
\label{subsec:disturbances_analysis}
In order to make a realistic simulation of the rotating motion of the spacecraft, the environment 
distrubances has to be taken into account. The preliminary study of these external torques is 
crucial for a realistic simulation. In the following paragraphs a brief introduction will be 
done for all the main disturbances, then the simulation of the specific satellite and orbit
will be presented, mainly to choose the two most relevant disturbances. This choice is reasonable 
since there are always two predominant effects of disturbance, while the other can be supposed small 
(usually some order of magnitude smaller, but always depends on the specific case) for this reason
all the disturbances are analyzed. 

\begin{table}[ht]
	\centering
	\begin{tabular}{|c|c|c|}
		\hline
		Type & Reference formula & Value [Nm] \\ \hline
		Gravity gradient & \( T_{\text{max}} = \frac{3G M_e}{2R^3} |I_{\text{max}} - I_{\text{min}}| \) & \\ \hline
		Solar radiation pressure & \( T_{\text{max}} = P_s S_{\text{max}} \left(1 + \frac{2}{3} \rho_r + \rho_d \right) (c_{ps} - c_g) \) & \\ \hline
		Magnetic field & \( T_{\text{max}} = |m_{\text{max}} B_{\text{max}}| \) & \\ \hline
		Aerodynamics & \( T_{\text{max}} = \frac{\rho v^2 S_{\text{max}} C_d}{2} (c_{pa} - c_g) \) & \\ \hline
	\end{tabular}
	\caption{Disturbances with solar panel retracted}
	\label{tab:torque_values}
\end{table}

\begin{table}[ht]
	\centering
	\begin{tabular}{|c|c|c|}
		\hline
		Type & Reference formula & Value [Nm] \\ \hline
		Gravity gradient & \( T_{\text{max}} = \frac{3G M_e}{2R^3} |I_{\text{max}} - I_{\text{min}}| \) & \\ \hline
		Solar radiation pressure & \( T_{\text{max}} = P_s S_{\text{max}} \left(1 + \frac{2}{3} \rho_r + \rho_d \right) (c_{ps} - c_g) \) & \\ \hline
		Magnetic field & \( T_{\text{max}} = |m_{\text{max}} B_{\text{max}}| \) & \\ \hline
		Aerodynamics & \( T_{\text{max}} = \frac{\rho v^2 S_{\text{max}} C_d}{2} (c_{pa} - c_g) \) & \\ \hline
	\end{tabular}
	\caption{Disturbances with solar panel extended}
	\label{tab:torque_values}
\end{table}




\subsubsection{Magnetic Disturbance}
\label{subsubsec:dist_mag}



\subsubsection{SRP Disturbance}
\label{subsubsec:dist_SRP}
SRP radiation torque is the disturbance generated by electromagnetic waves that impacts on
the spacecraft panels and generate a force. These forces acting on some of the panels could give rise to 
a net torque around the center of mass of the spacecraft. Only sun radiation will be considered 
in this case, a more deep analysis should consider infrared Earth radiation and reflected Earth radiation. 
In addition, no eclipse condition will be analyzed during all the simulation, a reasonable assumption for the 
sun-synchronous case orbit.

\subsubsection{Drag Disturbance}
\label{subsubsec:dist_drag}

Over extended periods, the spacecraft's engagement with the higher strata of Earth's atmosphere results
in the generation of a torque around its mass center. This influence may not be trivial. At altitudes 
less than 400 kilometers, the aerodynamic torque is the predominant factor, though its significance diminishes 
considerably beyond 700 kilometers altitude.

\[
T_{AERO} =
\begin{cases}
	\sum_{i=1}^{n} \vec{r}_i \times \vec{F}_i, & \text{if } \vec{N}_{bi} \cdot \vec{v}^{b}_{rel} \geq 0 \\
	0, & \text{if } \vec{N}_{bi} \cdot \vec{v}^{b}_{rel} < 0
\end{cases}
\text{ with }
\]

\[
\vec{F}_i = -\frac{1}{2} \rho C_D v_{rel}^2 \vec{v}_{rel}^{b} (\vec{N}_{bi} \cdot \vec{v}_{rel}^{b}) A_i \quad \text{n=number of faces}
\]


\subsubsection{Gravity Gradient Disturbance}

The gravity around the spacecraft is not uniform, hence a non-negligible torque will arise from there. Studying the torque generated by an elementary force acting on the elementary mass \( dm \) the equation for this is obtained:

\begin{equation*}
	dM = -\mathbf{r} \times \frac{Gm_t dm}{|\mathbf{R} + \mathbf{r}|^3} (\mathbf{R} + \mathbf{r})
\end{equation*}

Where \( \mathbf{r} \) is the distance of \( dm \) from the centre of mass and \( \mathbf{R} \) is the distance of the centre of mass from the centre of the Earth. Approximating this equation, and expressing the position vector of the centre of mass as the product of magnitude (\( R \)) with the direction cosines, it’s possible to centre this torque in the principal inertia axes. Integrating this equation the final form is achieved.

\begin{align*}
	M_x &= \frac{3Gm_t}{R^3} (I_z - I_y)c_2c_3 \\
	M_y &= \frac{3Gm_t}{R^3} (I_x - I_z)c_1c_3 \\
	M_z &= \frac{3Gm_t}{R^3} (I_y - I_x)c_1c_2
\end{align*}

The \( c_1, c_2 \) and \( c_3 \) are the direction cosines of the radial direction in the principal axes. Therefore if one of the principal axes is aligned with the radial direction the torque will be zero because only one of the direction cosines is non-zero.



\label{subsubsec:dist_GG}
