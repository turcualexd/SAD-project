\section{Environment}
\label{sec:environment}

\subsection{Satellite characterization }
\label{subsec:sat_characterization}


\subsection{Orbit characterization}
\label{subsec:orbit_characterization}

\subsection{Disturbances analysis}
\label{subsec:disturbances_analysis}
In order to make a realistic simulation of the rotating motion of the spacecraft, the environment 
distrubances has to be taken into account. The preliminary study of these external torques is 
crucial for a realistic simulation. In the following paragraphs a brief introduction will be 
done for all the main disturbances, then the simulation of the specific satellite and orbit
will be presented, mainly to choose the two most relevant disturbances. This choice is reasonable 
since there are always two predominant effects of disturbance, while the other can be supposed small 
(usually some order of magnitude smaller, but always depends on the specific case) for this reason
all the disturbances are analyzed. 

\begin{table}[ht]
	\centering
	\begin{tabular}{|c|c|c|}
		\hline
		Type & Reference formula & Value [Nm] \\ \hline
		Gravity gradient & \( T_{\text{max}} = \frac{3G M_e}{2R^3} |I_{\text{max}} - I_{\text{min}}| \) & \\ \hline
		Solar radiation pressure & \( T_{\text{max}} = P_s S_{\text{max}} \left(1 + \frac{2}{3} \rho_r + \rho_d \right) (c_{ps} - c_g) \) & \\ \hline
		Magnetic field & \( T_{\text{max}} = |m_{\text{max}} B_{\text{max}}| \) & \\ \hline
		Aerodynamics & \( T_{\text{max}} = \frac{\rho v^2 S_{\text{max}} C_d}{2} (c_{pa} - c_g) \) & \\ \hline
	\end{tabular}
	\caption{Disturbances with solar panel retracted}
	\label{tab:torque_values}
\end{table}

\begin{table}[ht]
	\centering
	\begin{tabular}{|c|c|c|}
		\hline
		Type & Reference formula & Value [Nm] \\ \hline
		Gravity gradient & \( T_{\text{max}} = \frac{3G M_e}{2R^3} |I_{\text{max}} - I_{\text{min}}| \) & \\ \hline
		Solar radiation pressure & \( T_{\text{max}} = P_s S_{\text{max}} \left(1 + \frac{2}{3} \rho_r + \rho_d \right) (c_{ps} - c_g) \) & \\ \hline
		Magnetic field & \( T_{\text{max}} = |m_{\text{max}} B_{\text{max}}| \) & \\ \hline
		Aerodynamics & \( T_{\text{max}} = \frac{\rho v^2 S_{\text{max}} C_d}{2} (c_{pa} - c_g) \) & \\ \hline
	\end{tabular}
	\caption{Disturbances with solar panel extended}
	\label{tab:torque_values}
\end{table}


\subsubsection{Magnetic Disturbance}
\label{subsubsec:dist_mag}



\subsubsection{SRP Disturbance}
\label{subsubsec:dist_SRP}
SRP radiation torque is the disturbance generated by electromagnetic waves that impacts on
the spacecraft panels and generate a force. These forces acting on some of the panels could give rise to 
a net torque around the center of mass of the spacecraft. Only sun radiation will be considered 
in this case, a more deep analysis should consider infrared Earth radiation and reflected Earth radiation. 
In addition, no eclipse condition will be analyzed during all the simulation, a reasonable assumption for the 
sun-synchronous case orbit.

\subsubsection{Drag Disturbance}
\label{subsubsec:dist_drag}

Over extended periods, the spacecraft's engagement with the higher strata of Earth's atmosphere results in the generation of a torque around its mass center. This influence may not be trivial. At altitudes less than 400 kilometers, the aerodynamic torque is the predominant factor, though its significance diminishes considerably beyond 700 kilometers altitude.

\[
T_{AERO} =
\begin{cases}
	\sum_{i=1}^{n} \vec{r}_i \times \vec{F}_i, & \text{if } \vec{N}_{bi} \cdot \vec{v}^{b}_{rel} \geq 0 \\
	0, & \text{if } \vec{N}_{bi} \cdot \vec{v}^{b}_{rel} < 0
\end{cases}
\text{ with }
\]

\[
\vec{F}_i = -\frac{1}{2} \rho C_D v_{rel}^2 \vec{v}_{rel}^{b} (\vec{N}_{bi} \cdot \vec{v}_{rel}^{b}) A_i \quad \text{n=number of faces}
\]


\subsubsection{Gravity Gradient Disturbance}

A generation of torque arises from the non-uniform gravity field across the spacecraft structure. The analytical formula for the Gravity Gradient torque is given as:

\[
M_x = \frac{3G M_e}{R^3} (I_z - I_y) c_2 c_3
\]

\[
M_y = \frac{3G M_e}{R^3} (I_x - I_z) c_1 c_3
\]

\[
M_z = \frac{3G M_e}{R^3} (I_y - I_x) c_1 c_2
\]

in the principle body axis. \( c_1, c_2 \) and \( c_3 \) are the direction cosines of the radial direction (that is the vector that unites the center of the earth and the satellite) in the principal axes.


\label{subsubsec:dist_GG}
