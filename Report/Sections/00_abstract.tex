\begin{abstract}
\addcontentsline{toc}{section}{Abstract}
\vspace*{5mm}

La presente relazione di prova finale intende dare una descrizione dell'endoreattore F-1 prodotto da Rocketdyne. Cinque di questi motori vennero installati sul primo stadio S-IC del vettore Saturn V che portò il primo uomo sulla luna. L'obiettivo di questo stadio era quello di portare il razzo ad una quota di 61 km, fornendo un $ \Delta v \simeq $ 2300 m/s. Questo primo requisito verrà mostrato attraverso un modello matematico che simula il volo dello stadio S-IC. \\
Di seguito verranno analizzati i principali sistemi per un singolo motore, partendo dal sistema di storaggio e alimentazione dei propellenti costituito dai serbatoi e dalla turbopompa, passando per il sistema di generazione di potenza che comprende il gas generator e la turbina. Passando dalla camera di combustione si arriva infine al sistema di espansione gasdinamico e allo studio del suo raffreddamento. Si provvederà inoltre a dare una descrizione qualitativa e quantitativa delle scelte progettuali applicate ai tempi.\\
La discussione dei processi di combustione del gas generator e della camera di spinta si basa su dati provenienti da simulazioni eseguite con i programmi CEAM e RPA. 

\end{abstract}