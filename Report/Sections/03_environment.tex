\section{Environment}
\label{sec:environment}

\subsection{Satellite characterization }
\label{subsec:sat_characterization}

\subsection{Orbit characterization}
\label{subsec:orbit_characterization}

\subsection{Disturbances analysis}
\label{subsec:disturbances_analysis}
In order to make a realistic simulation of the rotating motion of the spacecraft, the environment 
distrubances has to be taken into account. The preliminary study of these external torques is 
crucial for a realistic simulation. In the following paragraphs a brief introduction will be 
done for all the main disturbances, then the simulation of the specific satellite and orbit
will be presented, mainly to choose the two most relevant disturbances. This choice is reasonable 
since there are always two predominant effects of disturbance, while the other can be supposed small 
(usually some order of magnitude smaller, but always depends on the specific case) for this reason
all the disturbances are analyzed. 


\subsubsection{Magnetic Disturbance}
\label{subsubsec:dist_mag}



\subsubsection{SRP Disturbance}
\label{subsubsec:dist_SRP}
SRP radiation torque is the disturbance generated by electromagnetic waves that impacts on
the spacecraft panels and generate a force. These forces acting on some of the panels could give rise to 
a net torque around the center of mass of the spacecraft. Only sun radiation will be considered 
in this case, a more deep analysis should consider infrared Earth radiation and reflected Earth radiation. 
In addition, no eclipse condition will be analyzed during all the simulation, a reasonable assumption for the 
sun-synchronous case orbit.

\subsubsection{Drag Disturbance}
\label{subsubsec:dist_drag}



\subsubsection{Gravity Gradient Disturbance}
\label{subsubsec:dist_GG}