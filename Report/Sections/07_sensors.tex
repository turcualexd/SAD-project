\section{Sensors}
\label{sec:sensors}

Sensors are fundamental tools that allows the SC to know its orientation or angular velocity. Their presence onboard 
is fundamental for having a controlled motion of the satellite. In this section, the 3 sensors used will be presented, 
it will be also clarified the motivation that lead the team to choose two additional sensors over the horizon sensor 
assigned. 

\subsection{Horizon Sensor}
\label{subsec:horizon_sensor}

Horizon sensors are devices that can detect the centre of the planet, in our case Earth, and reconstruct the direction 
of that point with respect to the satellite. They usually work by analyzing the IR spectrum of the image through a thermopile
to reduce the visible light spectrum interference caused by transition of day and night on earth. 
Due to the nadir pointing requirment a static earth sensor has been chosen for this specific application. In particular, the 
\textit{Meisei Earth Horizon} was chosen, which has the following specifics:

\begin{table}[H]

    \centering
    \begin{tabular}{|c|c|c|}
    \hline
    $\bm{F.O.V. \, [\deg]}$ & $\bm{Accuracy \, [\deg]}$ & $\bm{Frequency \, [Hz]}$ \\
    \hline
    $33$ & $1$ & $30$  \\
    \hline
    \end{tabular}
    
    \caption{Real data for Horizon Sensor}
    \label{table:Hor_sensor}
    
\end{table}

Due to operational requirements, the static sensor has to point the Earth, in particular the optical
axis must have the same direction of the nadir (direction that links centre of earth and CoM of the S/C).
To fulfill this request, a good option could be to position the sensor on the face that also contains the 
payload, that is the face of the spacecraft main body (cube) that has normal along the $\boldsymbol{x_b}$ 
direction. The model implemented to simulate the behaviour of the sensor in the Simulink environment, 
was to take the real position of the S/C with respect to centre of Earth expressed in inertial space, change its
direction multiplying by -1 and express it in the $\mathcal{B}$ frame through the real attitude matrix $\boldsymbol{A_{B,N}}$. 
This unit vector is the input of the sensor block, here it is sampled through a zero-order hold of frequency specified by
\autoref{table:Hor_sensor} to simulate the digital nature of the sensor. Then some errors of measurments has to be added.
It was decided to model two typical effects of real sensor: the mounting error that cause misalignment and also an 
accuracy error modeled with a band-limited white noise on all the components of the direction vector.
\pagebreak
\subsection{Sun Sensor}
\label{subsec:sun_sensor}

Sun sensors are devices that detect the position of the Sun by measuring the incidence angle of its radiation on a sensor surface. This surface is typically made of materials that can generate a current proportional to the intensity of the incident light. From the measure of the current $I$ generated by the sensor surface of area $S$, knowing the intensity of the radiation $W$ and the coefficient $\alpha$ of the sensor, the angle of the incident light $\theta$ is internally computed as follows:

\begin{equation}
    I = \alpha S W \cos \theta  \implies  \theta = \arccos \left( \frac{I}{\alpha S W} \right)
\end{equation}

The unit vector pointing towards the Sun can be easily computed by the sensor using the knowledge of two angles obtained by placing two surfaces on different directions on the same plane.

As discussed in \autoref{sec:framework}, the satellite mission consists in pointing the Earth on a Sun-synchronous orbit with no eclipse periods. In view of this, it is important to prioritize accuracy over having the best Field of View (FOV) to select the sensor. For this reason, the choice falls upon a small and light Fine Sun sensor with good performances such as the \textit{AAC Clyde Space SS200}:

\begin{table}[H]

    \centering
    \begin{tabular}{|c|c|c|}
    \hline
    $\bm{F.O.V. \, [\deg]}$ & $\bm{Accuracy \, [\deg]}$ & $\bm{Frequency \, [Hz]}$ \\
    \hline
    $90$ & $0.3$ & $30$  \\
    \hline
    \end{tabular}
    
    \caption{Real data for Sun Sensor}
    \label{table:Sun_sensor}
    
\end{table}

This sensor has to be placed in the same direction as the solar panels, in the $\boldsymbol{z_b}$ direction, so the Sun is constantly visible (since the Orbit is Sun-synchronous). In order to not overcomplicate the Simulink model, it does not consider the FOV of the sensor, since in the detumbling manoeuvre the sensor is not used while in the tracking phase the Sun is always visible.
To simulate the output of the real sensor, the model takes the Sun direction computed summing up the initial Earth-Sun vector with the position vector of the satellite in the inertial frame, which is computed through the keplerian dynamics. This vector is then normalized, reversed and brought in the body frame using the attitude matrix. From here, errors have been added in the same way as described for the horizon sensor (\autoref{subsec:horizon_sensor}).