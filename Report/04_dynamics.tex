\section{Dynamics}
\label{sec:dynamics}
The equations of the dynamics rotating body motion used throughout the 
simulation are the Euler equations since rigid body motion assumption is made. 
The set of equations are referred to the principal axis frame 
of the satellite. This frame will be also called reference frame $\mathcal{B}$,
it is described by three unit vectors $\left\{ \boldsymbol{x_b, y_b, z_b,} \right\}$, 
that are in the direction of principal inertia axis. 

\begin{empheq}{equation*}
    \boldsymbol{I\dot{\omega}} + \boldsymbol{\omega} \times \boldsymbol{I\omega} = \boldsymbol{M_d} + \boldsymbol{M_c}
\end{empheq}

In the above equation the external torque has been divided in to 2 contributions, 
with clear distinction. $M_d$ describes the disturbance torques that act on the 
spacecraft due to environment and presented in the previous section, while $M_c$ 
is referred to the control torque that the actuators are generating to perform 
the tasks required by the control logic. \\
With particular reference to the Simulink model, two configurations of the satellite 
were considered: undeployed configuration (for detumbling phase) and extended configuration 
(for slew and pointing phases). As a consequence, the mass distribution and hence the inertia
matrix are different in terms of numerical values. This fact has been taken into account by implementing
a logic in the dynamic block of Simulink, that switches between the two matrices using a flag
based on the activation of the De-Tumbling control. This istantaneous switch is not completely realistic 
since the extraction of the panels would require some finite time, and in some way could influence the 
real dynamic of the satellite. Anyhow, for the microsat considered, the retracted configuration allows a 
faster detumbling, and also inertia loads and stresses are reduced on the solar panels.
